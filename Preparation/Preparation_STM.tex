\documentclass[a4paper]{article}

% --- Page layout and spacing ---
\usepackage[top=3cm, left=3.5cm, right=3.5cm, bottom=3cm]{geometry}
\usepackage[utf8]{inputenc}      % input encoding
\usepackage[T1]{fontenc}         % font encoding
\usepackage[english]{babel}
\usepackage{setspace}
\setlength{\parindent}{0pt}      % paragraph indentation
\setlength{\parskip}{0.8em}      % space between paragraphs
\setstretch{1.2}                 % line spacing
\usepackage{tocloft}             % section spacing in ToC
\setlength{\cftbeforesecskip}{10pt}
\setlength{\cftbeforesubsecskip}{4pt}
\usepackage{titlesec}            % section title spacing
\titlespacing*{\section}{0pt}{5.0ex plus 1ex minus .2ex}{1.0ex plus .2ex}
\titlespacing*{\subsection}{0pt}{3.0ex plus .5ex minus .2ex}{0.8ex plus .2ex}
\titlespacing*{\subsubsection}{0pt}{2.0ex plus .5ex minus .2ex}{0.8ex plus .2ex}

% --- Math and symbols ---
\usepackage{amsmath, amssymb}    % standard math
\usepackage{empheq}              % boxed equations etc.
\DeclareMathOperator{\artanh}{artanh}
\DeclareMathOperator{\sgn}{sgn}
\usepackage{bm}                  % bold math symbols
\usepackage{cancel}              % strikeout in math
\usepackage{siunitx}             % proper units
\renewcommand{\arraystretch}{0.7}

% --- Graphics and floats ---
\usepackage{graphicx}
\usepackage{float}
\usepackage{wrapfig}
\usepackage[justification=centering]{caption}
\usepackage{subcaption}
\captionsetup[figure]{font=small}

% --- Layout helpers ---
\usepackage{boxedminipage}
\usepackage{enumitem}
\usepackage{afterpage}
\usepackage{changepage}
\usepackage{pdfpages}           % include external PDFs
\usepackage{esvect}             % nice vector arrows
\usepackage{hyperref}           % hyperlinks

% --- Bibliography setup ---
\usepackage{csquotes}
\usepackage[backend=biber,style=numeric,sorting=none]{biblatex}
\addbibresource{references.bib}

% --- Fonts ---
\usepackage{lmodern}            % Computer Modern look across TeX distros



% --- Title page ---
\title{\textbf{Scanning Tunneling Microscope}}
\author{
  \\Preparation Report \\\\\\\\\\\\
  \textbf{Cem Boyaci} \\
  cemb93@zedat.fu-berlin.de \\\\\\
  \textbf{Javier Bellido Roldán}\\
  bellidoroj98@zedat.fu-berlin.de \\\\\\
  \textbf{Leon Goldammer} \\
  lg4278fu@zedat.fu-berlin.de \\\\
}
\date{}

\begin{document}
\maketitle
\thispagestyle{empty}

\section*{}
\begin{center}
\vspace{3cm}
Tutor: Vibhuti Rai \\[1cm]
\textbf{Fortgeschrittenenpraktikum, WS 2025/2026}\\
Berlin, 01.12.2025\\
Freie Universität Berlin\\
Fachbereich Physik
\end{center}



% --- Table of contents ---
\clearpage
\renewcommand*\contentsname{\huge Contents}
{
  \pagenumbering{gobble}
  \tableofcontents
  \clearpage
}
\pagenumbering{arabic}



% --- Introduction ---
\newpage
\setcounter{page}{1}

\section{Introduction}

The scanning tunneling microscope (STM) is a surface–analysis technique that enables real–space imaging of conductive materials with sub–nanometer resolution.
It operates by positioning a sharp metallic tip a few ångströms above the sample surface and detecting the tunneling current that flows when a bias voltage is applied.
Because this current depends exponentially on the tip–sample separation, the STM is extremely sensitive to atomic–scale height variations and can visualize both the topography and electronic structure of surfaces.
In this experiment we use a tabletop STM operated under ambient conditions to record and analyze surface images of highly oriented pyrolytic graphite (HOPG).
The aim is to become familiar with the operating principles of STM and to extract structural information from the acquired images.



% --- Physical Principles ---
\section{Physical Principles}

\subsection{Quantum Tunneling}

To understand how STM works it is useful to recall a few basic ideas from quantum mechanics.
Electrons in atoms occupy orbitals with specific energies, and their spatial distribution is described by a wavefunction.
A simple example is the particle-in-a-box model, where only certain standing-wave patterns are allowed.
At the edges of such a “box” the wavefunction decreases sharply, and in quantum mechanics it decays exponentially when it reaches a region where the electron cannot exist classically.
This behaviour also appears at the surface of a solid, where the periodic potential ends and the electron faces empty space.

Tunneling becomes relevant only when another conductive object is brought extremely close to the surface.
If the separation between the surface and the STM tip is only a few ångströms, the exponentially decaying wavefunctions of the two sides overlap (Fig.~\ref{fig:stm_wavefunction_overlap}).
In this situation an electron has a finite probability to appear on the other side of the gap, even though it cannot cross classically.
Because the probability depends exponentially on the separation, even tiny changes in distance cause large changes in the tunneling current \cite{STMAnleitung}.

\vspace{1em}
\begin{figure}[H]
\centering
\includegraphics[width=0.8\textwidth]{../resources/figures/STM_Tunneling_Schematic.png}
\caption{Schematic illustration of how tunneling becomes possible when the wavefunctions of the surface electrons $\psi_s$ and those of the STM tip $\psi_t$ overlap across the narrow gap~\cite{UW_AFM_STM_Workshop}.}
\label{fig:stm_wavefunction_overlap}
\end{figure}


\subsection{Electronic Structure of Solids and Graphite}

To understand which electrons contribute to tunneling we need to consider how electrons behave in solids.
When many atoms join to form a crystal, their atomic orbitals overlap and broaden into energy bands.
The highest fully occupied states form the valence band, while higher-energy states form the conduction band.
In conductors these bands either overlap slightly or touch at the Fermi level, so there are always electronic states available for motion.
The Fermi level is simply the energy that separates filled from unfilled states at low temperature.
Electrons close to this energy are only weakly bound and can respond easily to external influences.

Graphite is an example of such a conductor.
Each carbon atom uses three sp$^2$ hybrid orbitals to form strong in-plane $\sigma$ bonds, giving the hexagonal layer its mechanical stability.
The remaining $p_z$ orbital on each atom points perpendicular to the plane and overlaps with those of neighbouring atoms.
These $p_z$ orbitals form a delocalized $\pi$ band that sits at the Fermi level and is responsible for the electrical conductivity of graphite (Fig.~\Ref{fig:carbon_bonding}).
Because the $\pi$ electrons are the highest-energy occupied electrons, their wavefunctions reach the farthest out of the surface and dominate the tunneling process.

\vspace{1em}
\begin{figure}[H]
\centering
\includegraphics[width=0.8\textwidth]{../resources/figures/Carbon_Bonding.png}
\caption{Electronic orbitals in a graphene layer.
Each carbon atom forms three in-plane $\sigma$ bonds through sp$^2$ hybrid orbitals, while the $p_z$ orbitals extend perpendicular to the sheet and form the delocalized $\pi$ system.
These out-of-plane $\pi$ electrons dominate the tunneling current in STM \cite{proctor2019aps}.}
\label{fig:carbon_bonding}
\end{figure}

At the surface the $\pi$ electron cloud does not end abruptly but decreases gradually into the empty space.
The amount of electron density above a given atomic site depends on how the $p_z$ orbitals overlap, so the electronic structure above the surface is not uniform.
The quantity that describes these variations is the local density of states (LDOS).
It represents how many electronic states are available at a certain position and energy.

The STM is sensitive to this LDOS rather than to the atomic nuclei themselves.
Regions where the LDOS is higher have more electron density reaching toward the tip and therefore allow a stronger tunneling current.
This is why the STM image reflects the electronic structure of the surface, and why the delocalized $\pi$ electrons of graphite play such a central role in determining the contrast observed in atomic-resolution images.





% --- Experimental Setup ---
\section{Experimental Setup}

\subsection{xxx}

xxx



% --- Procedure ---
\section{Procedure}

\subsection{xxx}

xxx



% --- References ---
\newpage
\setstretch{1.0}
\printbibliography[heading=bibintoc]



\end{document}
