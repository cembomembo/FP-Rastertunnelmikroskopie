\documentclass[a4paper]{article}

% --- Page layout and spacing ---
\usepackage[top=3cm, left=3.5cm, right=3.5cm, bottom=3cm]{geometry}
\usepackage[utf8]{inputenc}      % input encoding
\usepackage[T1]{fontenc}         % font encoding
\usepackage[english]{babel}
\usepackage{setspace}
\setlength{\parindent}{0pt}      % paragraph indentation
\setlength{\parskip}{0.8em}      % space between paragraphs
\setstretch{1.2}                 % line spacing
\usepackage{tocloft}             % section spacing in ToC
\setlength{\cftbeforesecskip}{10pt}
\setlength{\cftbeforesubsecskip}{4pt}
\usepackage{titlesec}            % section title spacing
\titlespacing*{\section}{0pt}{5.0ex plus 1ex minus .2ex}{1.0ex plus .2ex}
\titlespacing*{\subsection}{0pt}{3.0ex plus .5ex minus .2ex}{0.8ex plus .2ex}
\titlespacing*{\subsubsection}{0pt}{2.0ex plus .5ex minus .2ex}{0.8ex plus .2ex}

% --- Math and symbols ---
\usepackage{amsmath, amssymb}    % standard math
\usepackage{empheq}              % boxed equations etc.
\DeclareMathOperator{\artanh}{artanh}
\DeclareMathOperator{\sgn}{sgn}
\usepackage{bm}                  % bold math symbols
\usepackage{cancel}              % strikeout in math
\usepackage{siunitx}             % proper units
\renewcommand{\arraystretch}{0.7}

% --- Graphics and floats ---
\usepackage{graphicx}
\usepackage{float}
\usepackage{wrapfig}
\usepackage[justification=centering]{caption}
\usepackage{subcaption}
\captionsetup[figure]{font=small}

% --- Layout helpers ---
\usepackage{boxedminipage}
\usepackage{enumitem}
\usepackage{afterpage}
\usepackage{changepage}
\usepackage{pdfpages}           % include external PDFs
\usepackage{esvect}             % nice vector arrows
\usepackage{hyperref}           % hyperlinks

% --- Bibliography setup ---
\usepackage{csquotes}
\usepackage[backend=biber,style=numeric,sorting=none]{biblatex}
\addbibresource{references.bib}

% --- Fonts ---
\usepackage{lmodern}            % Computer Modern look across TeX distros



% --- Title page ---
\title{\textbf{Scanning Tunneling Microscope}}
\author{
  \\Preparation Report \\\\\\\\\\\\
  \textbf{Cem Boyaci} \\
  cemb93@zedat.fu-berlin.de \\\\\\
  \textbf{Javier Bellido Roldán}\\
  bellidoroj98@zedat.fu-berlin.de \\\\\\
  \textbf{Leon Goldammer} \\
  lg4278fu@zedat.fu-berlin.de \\\\
}
\date{}

\begin{document}
\maketitle
\thispagestyle{empty}

\section*{}
\begin{center}
\vspace{3cm}
Tutor: Vibhuti Rai \\[1cm]
\textbf{Fortgeschrittenenpraktikum, WS 2025/2026}\\
Berlin, 01.12.2025\\
Freie Universität Berlin\\
Fachbereich Physik
\end{center}



% --- Table of contents ---
\clearpage
\renewcommand*\contentsname{\huge Contents}
{
  \pagenumbering{gobble}
  \tableofcontents
  \clearpage
}
\pagenumbering{arabic}



% --- Introduction ---
\newpage
\setcounter{page}{1}

\section{Introduction}

The scanning tunneling microscope (STM) is a surface–analysis technique that enables real–space imaging of conductive materials with sub–nanometer resolution.
It operates by positioning a sharp metallic tip a few ångströms above the sample surface and detecting the tunneling current that flows when a bias voltage is applied.
Because this current depends exponentially on the tip–sample separation, the STM is extremely sensitive to atomic–scale height variations and can visualize both the topography and electronic structure of surfaces.
In this experiment we use a tabletop STM operated under ambient conditions to record and analyze surface images of highly oriented pyrolytic graphite (HOPG).
The aim is to become familiar with the operating principles of STM and to extract structural information from the acquired images.



% --- Physical Principles ---
\section{Physical Principles}

\subsection{Quantum Tunneling}

To understand how STM works it is useful to recall a few basic ideas from quantum mechanics.
Electrons in atoms occupy orbitals with specific energies, and their spatial distribution is described by a wavefunction.
A simple example is the particle-in-a-box model, where only certain standing-wave patterns are allowed.
At the edges of such a “box” the wavefunction decreases sharply, and in quantum mechanics it decays exponentially when it reaches a region where the electron cannot exist classically.
This behaviour also appears at the surface of a solid, where the periodic potential ends and the electron faces empty space.

Tunneling becomes relevant only when another conductive object is brought extremely close to the surface.
If the separation between the surface and the STM tip is only a few ångströms, the exponentially decaying wavefunctions of the two sides overlap (Fig.~\ref{fig:stm_wavefunction_overlap}).
In this situation an electron has a finite probability to appear on the other side of the gap, even though it cannot cross classically.
Because the probability depends exponentially on the separation, even tiny changes in distance cause large changes in the tunneling current \cite{STMAnleitung}.

\vspace{1em}
\begin{figure}[H]
\centering
\includegraphics[width=0.8\textwidth]{../resources/figures/STM_Tunneling_Schematic.png}
\caption{Schematic illustration of how tunneling becomes possible when the wavefunctions of the surface electrons $\psi_s$ and those of the STM tip $\psi_t$ overlap across the narrow gap~\cite{UW_AFM_STM_Workshop}.}
\label{fig:stm_wavefunction_overlap}
\end{figure}


\subsection{Electronic Structure of Solids and Graphite}

To understand which electrons contribute to tunneling we need to consider how electrons behave in solids.
When many atoms join to form a crystal, their atomic orbitals overlap and broaden into energy bands.
The highest fully occupied states form the valence band, while higher-energy states form the conduction band.
In conductors these bands either overlap slightly or touch at the Fermi level, so there are always electronic states available for motion.
The Fermi level is simply the energy that separates filled from unfilled states at low temperature.
Electrons close to this energy are only weakly bound and can respond easily to external influences.

Graphite is an example of such a conductor.
Each carbon atom uses three sp$^2$ hybrid orbitals to form strong in-plane $\sigma$ bonds, giving the hexagonal layer its mechanical stability.
The remaining $p_z$ orbital on each atom points perpendicular to the plane and overlaps with those of neighbouring atoms.
These $p_z$ orbitals form a delocalized $\pi$ band that sits at the Fermi level and is responsible for the electrical conductivity of graphite (Fig.~\Ref{fig:carbon_bonding}).
Because the $\pi$ electrons are the highest-energy occupied electrons, their wavefunctions reach the farthest out of the surface and dominate the tunneling process.

\vspace{1em}
\begin{figure}[H]
\centering
\includegraphics[width=0.8\textwidth]{../resources/figures/Carbon_Bonding.png}
\caption{Electronic orbitals in a graphene layer.
Each carbon atom forms three in-plane $\sigma$ bonds through sp$^2$ hybrid orbitals, while the $p_z$ orbitals extend perpendicular to the sheet and form the delocalized $\pi$ system.
These out-of-plane $\pi$ electrons dominate the tunneling current in STM \cite{proctor2019aps}.}
\label{fig:carbon_bonding}
\end{figure}

At the surface the $\pi$ electron cloud does not end abruptly but decreases gradually into the empty space.
The amount of electron density above a given atomic site depends on how the $p_z$ orbitals overlap, so the electronic structure above the surface is not uniform.
The quantity that describes these variations is the local density of states (LDOS).
It represents how many electronic states are available at a certain position and energy.

The STM is sensitive to this LDOS rather than to the atomic nuclei themselves.
Regions where the LDOS is higher have more electron density reaching toward the tip and therefore allow a stronger tunneling current.
This is why the STM image reflects the electronic structure of the surface, and why the delocalized $\pi$ electrons of graphite play such a central role in determining the contrast observed in atomic-resolution images.


\subsection{Principle of STM Operation}

Once the tip is brought close enough for tunneling to occur, a bias voltage is applied between the tip and the sample.
This voltage shifts their Fermi levels relative to each other, allowing electrons to tunnel from the filled states of one side into the empty states of the other.
The tunneling current depends extremely sensitively on the tip–sample separation $z$ and can be written in the approximate form
\[
I_t \sim V_t \exp\!\big(-c\,\sqrt{\Phi}\, z\big),
\]
where $V_t$ is the applied bias voltage, $\Phi$ is an effective barrier height (in eV), $z$ is the tip–sample distance (in \AA), and $c \approx 1.02\,\text{\AA}^{-1}\,\text{eV}^{-1/2}$ collects the fundamental constants and unit conversions~\cite{STMAnleitung}.
Because of the exponential dependence on $z$, even sub-ångström changes in distance lead to large variations in the tunneling current.

To record an image, the STM moves the tip laterally while either keeping the height fixed or adjusting it continuously, as illustrated in Fig.~\ref{fig:stm_scan_modes}.
In constant–height mode the tip position $z$ remains fixed, and contrast arises solely from variations in the tunneling current.
This mode is fast but only safe on atomically flat surfaces, since the tip may collide with protrusions on the sample.

In constant–current mode, which is used in this experiment, the STM continuously adjusts the tip height so that the tunneling current matches a chosen setpoint.
A feedback loop monitors the current in real time and moves the tip up or down using a piezoelectric actuator whenever the measured current deviates from this value.
The vertical motion required to maintain constant current is then recorded at each lateral scan position.

\vspace{1em}
\begin{figure}[H]
\centering
\includegraphics[width=1.0\textwidth]{../resources/figures/STM_Modes.png}
\caption{Basic working principle of STM imaging.
(a) Constant–height mode, where the tip follows a fixed distance while the tunneling current varies with the surface, which risks tip crash on rough samples.
(b) Constant–current mode, where a feedback loop adjusts the tip height to maintain a constant current.
The recorded height forms the STM image~\cite{huda2016thesis}.}
\label{fig:stm_scan_modes}
\end{figure}

Because the measurement relies on maintaining a constant tunneling probability, the STM image essentially represents the height at which the electronic density accessible for tunneling matches the selected setpoint.
This explains why STM images are strongly influenced by the electronic structure of the surface and not solely by its geometric topography.


\subsection{What the STM Actually Maps}

The previous subsections showed that the tunneling current depends on the overlap between tip and sample wavefunctions and is therefore extremely sensitive to the local electronic structure above the surface.
To understand what the STM image represents, it is necessary to look more closely at the quantity that determines this overlap: the local density of states (LDOS).

The LDOS describes how many electronic states are available at a particular energy and spatial position.
Unlike the total density of states of a solid, which counts all states in the bulk, the LDOS resolves how strongly each state contributes at a specific point in space.
At the surface this quantity is highly non-uniform, even for materials with perfectly regular atomic arrangements.

Which part of the LDOS contributes to the tunneling current depends on the applied bias voltage.
For small biases the main contribution comes from states very close to the Fermi level, because only these states lie within the narrow energy window through which electrons can tunnel.
States far below the Fermi level are fully occupied and do not participate, while states far above it are empty and cannot accept electrons.
This energy selectivity explains why STM is sensitive to the electronic structure in a narrow region around the Fermi energy rather than to all electrons in the material.
An overview of the different STM measurement modes derived from the LDOS is shown in Fig.~\ref{fig:stm_ldos_overview}.

\vspace{1em}
\begin{figure}[H]
\centering
\includegraphics[width=0.92\textwidth]{../resources/figures/STM_Measurement_Types.png}
\caption{Different ways in which the STM can probe the local density of states (LDOS).
The LDOS depends on spatial position $(x,y)$ and energy $E$, forming a three-dimensional dataset (a).
From this dataset one can extract topographic images (b), LDOS maps at a fixed energy (c), tunneling spectra $dI/dV$ at a single point (d), or linecuts showing how the LDOS varies along a spatial path (e) \cite{hoffmanSTM}.}
\label{fig:stm_ldos_overview}
\end{figure}

The connection between tunneling current and LDOS was established by Tersoff and Hamann, who showed that the tunneling current is approximately proportional to the sample LDOS at the tip position and at the relevant energy~\cite{tersoff1985prb}.
This clarified a crucial point: STM does not directly map atomic nuclei.
Instead, it images the spatial variation of the LDOS, which may or may not resemble the underlying geometric lattice.

This distinction was essential in early STM studies.
Several early images exhibited contrast patterns that could not be explained by assuming that the tip traced the nuclear positions of surface atoms.
Only when the LDOS-based interpretation was adopted did these inconsistencies become understandable.
For some materials the LDOS follows the atomic lattice closely, while for others the electronic structure differs significantly from the geometric arrangement.




% --- Experimental Setup ---
\section{Experimental Setup}

\subsection{xxx}

xxx



% --- Procedure ---
\section{Procedure}

\subsection{xxx}

xxx



% --- References ---
\newpage
\setstretch{1.0}
\printbibliography[heading=bibintoc]

\section*{Author's Note}
AI-based writing and programming tools were used in a supporting role to refine the wording of this report and to assist in formatting Python and LaTeX code.
All scientific analysis, data evaluation, and interpretation were carried out independently by the authors.


\end{document}
