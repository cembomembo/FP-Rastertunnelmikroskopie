\documentclass[a4paper]{article}

% --- Page layout and spacing ---
\usepackage[top=3cm, left=3.5cm, right=3.5cm, bottom=3cm]{geometry}
\usepackage[utf8]{inputenc}      % input encoding
\usepackage[T1]{fontenc}         % font encoding
\usepackage[english]{babel}
\usepackage{setspace}
\setlength{\parindent}{0pt}      % paragraph indentation
\setlength{\parskip}{0.8em}      % space between paragraphs
\setstretch{1.2}                 % line spacing
\usepackage{tocloft}             % section spacing in ToC
\setlength{\cftbeforesecskip}{10pt}
\setlength{\cftbeforesubsecskip}{4pt}
\usepackage{titlesec}            % section title spacing
\titlespacing*{\section}{0pt}{5.0ex plus 1ex minus .2ex}{1.0ex plus .2ex}
\titlespacing*{\subsection}{0pt}{3.0ex plus .5ex minus .2ex}{0.8ex plus .2ex}
\titlespacing*{\subsubsection}{0pt}{2.0ex plus .5ex minus .2ex}{0.8ex plus .2ex}

% --- Math and symbols ---
\usepackage{amsmath, amssymb}    % standard math
\usepackage{empheq}              % boxed equations etc.
\DeclareMathOperator{\artanh}{artanh}
\DeclareMathOperator{\sgn}{sgn}
\usepackage{bm}                  % bold math symbols
\usepackage{cancel}              % strikeout in math
\usepackage{siunitx}             % proper units
\renewcommand{\arraystretch}{0.7}

% --- Graphics and floats ---
\usepackage{graphicx}
\usepackage{float}
\usepackage{wrapfig}
\usepackage[justification=centering]{caption}
\usepackage{subcaption}
\captionsetup[figure]{font=small}

% --- Layout helpers ---
\usepackage{boxedminipage}
\usepackage{enumitem}
\usepackage{afterpage}
\usepackage{changepage}
\usepackage{pdfpages}           % include external PDFs
\usepackage{esvect}             % nice vector arrows
\usepackage{hyperref}           % hyperlinks

% --- Bibliography setup ---
\usepackage{csquotes}
\usepackage[backend=biber,style=numeric,sorting=none]{biblatex}
\addbibresource{references.bib}

% --- Fonts ---
\usepackage{lmodern}            % Computer Modern look across TeX distros



% --- Title page ---
\title{\textbf{Scanning Tunneling Microscope}}
\author{
  \\Evaluation Report \\\\\\\\\\\\
  \textbf{Cem Boyaci} \\
  cemb93@zedat.fu-berlin.de \\\\\\
  \textbf{Javier Bellido Roldán}\\
  bellidoroj98@zedat.fu-berlin.de \\\\\\
  \textbf{Leon Goldammer} \\
  lg4278fu@zedat.fu-berlin.de \\\\
}
\date{}

\begin{document}
\maketitle
\thispagestyle{empty}

\section*{}
\begin{center}
\vspace{3cm}
Tutor: Vibhuti Rai \\[1cm]
\textbf{Fortgeschrittenenpraktikum, WS 2025/2026}\\
Berlin, 03.12.2025\\
Freie Universität Berlin\\
Fachbereich Physik
\end{center}



% --- Table of contents ---
\clearpage
\renewcommand*\contentsname{\huge Contents}
{
  \pagenumbering{gobble}
  \tableofcontents
  \clearpage
}
\pagenumbering{arabic}



% --- Introduction ---
\newpage
\setcounter{page}{1} 

\section{Introduction}

The scanning tunneling microscope (STM) is a surface–analysis technique that enables real–space imaging of conductive materials with sub–nanometer resolution.
It operates by positioning a sharp metallic tip a few ångströms above the sample surface and detecting the tunneling current that flows when a bias voltage is applied.
Because this current depends exponentially on the tip–sample separation, the STM is extremely sensitive to atomic–scale height variations and can visualize both the topography and electronic structure of surfaces.
In this experiment we use a tabletop STM operated under ambient conditions to record and analyze surface images of highly oriented pyrolytic graphite (HOPG).
The aim is to become familiar with the operating principles of STM and to extract structural information from the acquired images.



% --- Physical Principles ---
\section{Physical Principles}

\subsection{xxx}

xxx



% --- Experimental Setup ---
\section{Experimental Setup}

\subsection{xxx}

xxx



% --- Procedure ---
\section{Procedure}

\subsection{xxx}

xxx



% --- Results ---
\section{Results}

\subsection{xxx}

xxx



% --- Discussion ---
\section{Discussion}

\subsection{xxx}

xxx



% --- Conclusion ---
\section{Conclusion}

xxx



% --- Appendix ---
\newpage
\section{Appendix}

Additional files:

\begin{itemize}
  \item \texttt{"Python\_STM.zip"}
  \item \texttt{"LabReport\_STM.pdf"}
\end{itemize}



% --- References ---
\setstretch{1.0}
\printbibliography[heading=bibintoc]

\section*{Author's Note}
AI-based writing and programming tools were used in a supporting role to refine the wording of this report and to assist in formatting Python and LaTeX code.
All scientific analysis, data evaluation, and interpretation were carried out independently by the authors.



% --- Lab Report ---

\newpage
%\includepdf[pages=-, scale=0.9, pagecommand={\thispagestyle{empty}}]{../resources/LabReport_STM.pdf}


\end{document}
